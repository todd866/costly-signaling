\documentclass[11pt]{letter}
\usepackage[margin=1in]{geometry}
\usepackage[hidelinks]{hyperref}

\signature{Ian Todd\\Sydney Medical School\\University of Sydney\\itod2305@uni.sydney.edu.au}

\address{Ian Todd\\Sydney Medical School\\University of Sydney\\Sydney, NSW 2006, Australia}

\begin{document}

\begin{letter}{Editorial Office\\Biology \& Philosophy}

\opening{Dear Editors,}

I am submitting ``Costly Signaling and Coalition Formation Across Biological Scales'' for consideration as an original article in \textit{Biology \& Philosophy}.

This manuscript is a conceptual sequel to my paper ``Power as Control of Controllers: A Cross-Scale Theory of Agency,'' currently under review at \textit{Biology \& Philosophy}. That paper developed a cross-scale account of agency and power in biological systems. It established continuity from microbes to humans but left open a prior question: how do coalitions capable of exercising power form in the first place?

The present paper addresses this gap. I argue that costly signaling provides a general solution to coalition formation across biological scales. When coordination stakes are high, agents face a commitment verification problem: how can potential cooperators distinguish genuine allies from defectors before undertaking costly collective action? Signals that are expensive to produce---and differentially costly for defectors---create a separating equilibrium.

The paper offers a \textbf{design-space explanation}, not a genealogical one. I do not claim that human religions descended from bacterial quorum sensing. The claim is convergence: under shared stability constraints (high-stakes coordination, partial observability, defection incentives), viable solutions cluster in the same region of design space. The structural features often labeled ``religious''---ritual, sacred markers, deviation punishment, identity fusion---are attractors in the space of coalition-stabilizing architectures.

The paper engages with the existing literature on costly signaling in religion (Irons, Iannaccone, Berman, Sosis) and extends it in two directions: (1) generalizing beyond religion to coalition formation as a cross-scale design constraint, and (2) providing a unified treatment across biological scales from microbial quorum sensing through multicellular coordination to ideological movements.

Key contributions include:
\begin{itemize}
\item A formal account of why defectors pay higher signaling costs than loyalists
\item Cross-scale analysis from microbes through primates to human ideological movements
\item Testable predictions: evidence against beliefs can strengthen commitment; internal deviation is punished more than external opposition; threatened coalitions escalate signal demands
\item Agent-based simulations demonstrating the separating equilibrium, cost ratchet dynamics, and cellular coalition maintenance
\end{itemize}

I believe this paper is well suited to \textit{Biology \& Philosophy} because it engages directly with debates on functional explanation, levels of selection, and cross-scale continuity in biology---questions central to the journal's scope. The manuscript advances a constraint-based account of coalition individuality and norm-governed commitment verification.

The manuscript is approximately 7,500 words and has not been submitted elsewhere. I have no conflicts of interest to declare.

Thank you for considering this submission.

\closing{Sincerely,}

\end{letter}
\end{document}
